\documentclass[a4paper,11pt,twoside]{book}
\usepackage[BW]{optional}
\usepackage{skyrelisasy}
\begin{document}
\section{skyreliopavadinimas} %Įrašykite skyrelio pavadinimą.

%Rašyti čia

%Testai --------------------------------------------------------------
%Lietuvių kalbos testas. Sukompiliavę turite matyti užrašą ``Teorema''
%ir taisyklingas lietuviškas raides bei kabutes.  Jei viskas veikia,
%ištrinkite šį komentarą ir žemiau esančias tris eilutes.

\begin{thm} 
  ąčęėįšųū„“ž
\end{thm}

%Asymptote testas. Sukompiliavę skyrelisasy.tex su latex,
%skyrelisasy-1.asy su asymptote ir skyrelisasy.tex vėl su latex,
%turite matyti brėžinį.  Jei viskas veikia, ištrinkite šį komentarą ir
%žemiau esančias eilutes. 

\begin{center}
\begin{asy}
import olympiad;
size(200);
pair A, B, C, D, E, F, G, H;
path a1, a2;
A=origin; B=(100,0);
a1=circle(A,40);
a2=circle(B,40);
draw(a1);
draw(a2);
dot(A,blue);
dot(B,blue);
C=waypoint(a1,-0.10);
D=waypoint(a1,-0.40);
E=waypoint(a1,0.20);
F=waypoint(a2,0.1);
G=waypoint(a2,0.2);
H=waypoint(a2,0.44);
add(anglem(D,A,C,300,green,0));
add(anglem(D,E,C,300,green,0));
add(anglem(H,G,F,300,blue,0));
add(anglem(H,B,F,300,blue,0));
label("$\beta$",E,1.5*(-0.4,-2),deepgreen);
label("$2\beta$",A,1.5*(0,-2),deepgreen);
label("$\alpha$",G,1.5*(0,-2),deepblue);
label("$2\alpha$",B,1.5*(0,-2),deepblue);
draw(A--C--E--D--cycle);
draw(C--D);
draw(B--F--G--H--cycle);
draw(F--H);
dot(A,blue);
dot(B,blue);
dot(C,blue);
dot(D,blue);
dot(E,blue);
dot(F,blue);
dot(G,blue);
dot(H,blue);
\end{asy}
\end{center}
%--------------------------------------------------------------------


\end{document}
